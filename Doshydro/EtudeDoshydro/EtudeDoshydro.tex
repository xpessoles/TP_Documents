\documentclass[10pt]{article}
\input{style/coursHeadings}
\input{style/programHeadings}
\input{style/macros_SII}
\input{style/macros_Titres}
\input{style/macros_Frames}

%Si le boolen xp est vrai : compilation pour xabi
%Sinon compilation Damien
\newboolean{xp}
\setboolean{xp}{true}

\newboolean{prof}
\setboolean{prof}{true}

\usepackage[%
    pdftitle={Étude complète de la pompe Doshydro},
    pdfauthor={Xavier Pessoles},
    colorlinks=true,
    linkcolor=blue,
    citecolor=magenta]{hyperref}

\newif\ifprof
\proftrue
%\proffalse

\newif\iftd
\tdtrue
%\tdfalse

\def\discipline{Sciences Industrielles de l'Ingénieur}
\def\xxtitre{\ifthenelse{\boolean{xp}}{
Étude des systèmes de laboratoire}{
Chapitre  -- }}

\def\xxsoustitre{\ifthenelse{\boolean{xp}}{
Pompe Doshydro}{
Partie  -- }}

\def\xxauteur{\ifthenelse{\boolean{xp}}{
Xavier \textsc{Pessoles}}{}}

\def\xxpied{\ifthenelse{\boolean{xp}}{
\textit{Étude des systèmes de laboratoire}\\
\textit{Maxpid}}{
\xxtitre}}

\def\xxcathegorie{\ifthenelse{\boolean{xp}}{
2013 -- 2014 \\
Xavier \textsc{Pessoles}}{
Informatique - Cours}}





%---------------------------------------------------------------------------


\begin{document}

\ifthenelse{\boolean{xp}}{\input{style/enteteXP}}{\input{style/enteteDI}}

\begin{minipage}[b]{.3\linewidth}
\begin{center}
\includegraphics[width=.95\linewidth]{images/Doshydro_labo}

\textit{Système pédagogique}
\end{center}
\end{minipage} \hfill
\begin{minipage}[b]{.3\linewidth}
\begin{center}
\includegraphics[width=.95\linewidth]{images/Doshydro_SW}

\textit{Représentation 3D du système}
\end{center}
\end{minipage} \hfill
\begin{minipage}[b]{.3\linewidth}
\begin{center}
%\includegraphics[width=.75\linewidth]{images/CroixMalte_3d}

%\textit{Représentation 3D de la Croix de Malte}
\end{center}
\end{minipage}



\setlength{\parskip}{0ex plus 0.2ex minus 0ex}
 \renewcommand{\contentsname}{}
 \renewcommand{\baselinestretch}{1}

\tableofcontents

 \renewcommand{\baselinestretch}{1.2}
\setlength{\parskip}{2ex plus 0.5ex minus 0.2ex}



\section{Modélisation cinématique de la pompe}
\subsection{Schéma cinématique}


\begin{center}
 \includegraphics[width=.95\textwidth]{images/SchemaCinematique}
\end{center}

On a : 
\begin{itemize}
\item $\vect{O_0 O_2} = R \vect{x_2}$ avec $R= \text{mm}$;
\item $\vect{O_2 A} = \lambda(t) \vect{y_0}$;
\item $\vect{BA} = a \vect{x_0}+b \vect{y_0}$ avec $a= \text{mm}$ et $b= \text{mm}$;
\item $\vect{B O_0} = \mu(t) \vect{x_0}$;
\item la vis a $n$ filets;
\item la roue a $Z$ dents. 
\end{itemize}
\subsection{Détermination de la loi Entrée / Sortie}
On cherche d'abord à établir la loi entre la rotation de la pièce 2 ($\varphi$) et la translation du piston ($\mu$) . On peut écrire la fermeture de chaîne suivante : 
\begin{eqnarray*}
\vect{O_0 O_2} + \vect{O_2 A} + \vect{AB} + \vect{B O_0} &= &\vect{0}\\
\Longleftrightarrow 
 R \vect{x_2} + \lambda(t) \vect{y_0} - a \vect{x_0} - b \vect{y_0} + \mu(t) \vect{x_0} &=& \vect{0}  \\
 \Longleftrightarrow
 R \left(\cos \varphi(t) \vect{x_0} + \sin\varphi(t) \vect{y_0} \right) + \lambda(t) \vect{y_0} - a \vect{x_0} - b \vect{y_0} + \mu(t) \vect{x_0} &=& \vect{0}
\end{eqnarray*}

Grâce à la projection sur $\vect{x_0}$ on obtient directement : 
$$
 R \cos \varphi (t) - a + \mu(t)  = 0 \Longleftrightarrow
 \mu(t)= a - R \cos\varphi (t)
$$

On peut alors exprimer la position du piston en fonction de la position angulaire du moteur : 
$$  \mu(t)= a - R \cos\left(\dfrac{n}{Z} \cdot \theta (t)\right) $$

\subsection{Détermination de la loi en vitesse}
$$
 \dfrac{d\mu(t)}{dt}=  R  \dfrac{d\varphi(t)}{dt}\sin\varphi (t) 
 \quad \text{et} \quad
 \dfrac{d\mu(t)}{dt}=  R  \dfrac{n}{Z} \dfrac{d\theta(t)}{dt}\sin\theta (t) 
$$

\subsection{Tracé des courbes} 

\begin{thebibliography}{2}
\bibitem{xx}{xx}
\end{thebibliography}
\end{document}


