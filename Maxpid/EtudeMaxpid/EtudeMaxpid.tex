\documentclass[10pt]{article}
\input{style/coursHeadings}
\input{style/programHeadings}
\input{style/macros_SII}
\input{style/macros_Titres}
\input{style/macros_Frames}

%Si le boolen xp est vrai : compilation pour xabi
%Sinon compilation Damien
\newboolean{xp}
\setboolean{xp}{true}

\newboolean{prof}
\setboolean{prof}{true}

\usepackage[%
    pdftitle={Étude complète du Maxpid},
    pdfauthor={Xavier Pessoles},
    colorlinks=true,
    linkcolor=blue,
    citecolor=magenta]{hyperref}


\def\discipline{Sciences Industrielles de l'Ingénieur}
\def\xxtitre{\ifthenelse{\boolean{xp}}{
Étude des systèmes de laboratoire}{
Chapitre  -- }}

\def\xxsoustitre{\ifthenelse{\boolean{xp}}{
Bras robotisé Maxpid}{
Partie  -- }}

\def\xxauteur{\ifthenelse{\boolean{xp}}{
Xavier \textsc{Pessoles}}{}}

\def\xxpied{\ifthenelse{\boolean{xp}}{
Étude des systèmes de laboratoire\\
Maxpid}{
\xxtitre}}

\def\xxcathegorie{\ifthenelse{\boolean{xp}}{
2013 -- 2014 \\
Xavier \textsc{Pessoles}}{
Informatique - Cours}}





%---------------------------------------------------------------------------


\begin{document}

\ifthenelse{\boolean{xp}}{\input{style/enteteXP}}{\input{style/enteteDI}}

\begin{minipage}[b]{.3\linewidth}
\begin{center}
%\includegraphics[width=.95\linewidth]{images/capsuleuse_ph}

\textit{Système pédagogique}
\end{center}
\end{minipage} \hfill
\begin{minipage}[b]{.3\linewidth}
\begin{center}
%\includegraphics[width=.95\linewidth]{images/capsuleuse_3d}

\textit{Représentation 3D du système}
\end{center}
\end{minipage} \hfill
\begin{minipage}[b]{.3\linewidth}
\begin{center}
%\includegraphics[width=.75\linewidth]{images/CroixMalte_3d}

%\textit{Représentation 3D de la Croix de Malte}
\end{center}
\end{minipage}



\setlength{\parskip}{0ex plus 0.2ex minus 0ex}
 \renewcommand{\contentsname}{}
 \renewcommand{\baselinestretch}{1}

\tableofcontents

 \renewcommand{\baselinestretch}{1.2}
\setlength{\parskip}{2ex plus 0.5ex minus 0.2ex}



\section{Modélisation cinématique du bras Maxpid}
\subsection{Schéma cinématique}


\begin{center}
 \includegraphics[width=.95\textwidth]{images/schema}
\end{center}

\subsection{Détermination de la loi Entrée / Sortie}

La fermeture de chaîne cinématique s'écrit ainsi : 
$$
\vect{AC}+\vect{CD}+\vect{DA}=\vect{0} 
\Longleftrightarrow
\lambda\vect{x_1} - b \vect{x_4} - a \vect{x_0'} = \vect{0}
$$

Projetons cette relation dans le repère $\mathcal{R}_{0'}$ :
$$
\lambda\left(\cos\beta \vect{x_0'}+\sin \beta \vect{y_0'} \right) 
- b \left(\cos\theta' \vect{x_0'}+\sin \theta' \vect{y_0'} \right) - a \vect{x_0'} = \vect{0}
\Longrightarrow 
\left\{
\begin{array}{l}
\lambda\cos\beta - b \cos\theta' - a  = 0 \\
\lambda\sin \beta - b \sin \theta' = 0
\end{array}
\right.
$$

Pour exprimer la loi entrée sortie, commençons par déterminer $\theta'$ en fonction de $\lambda$ : 
$$
\lambda^2 
=  a^2 +b^2\cos^2\theta' + 2ab\cos\theta'  + b^2\sin^2 \theta'
=  a^2 +b^2 + 2ab\cos\theta' 
\Longleftrightarrow 
\theta' = \arccos \left(\dfrac{\lambda^2 - a^2 - b^2 }{2ab} \right)
$$

Une fermeture angulaire nous permet d'exprimer $\theta$ : $\theta' = \alpha + \theta$, on a donc :
$$
\theta = \arccos \left(\dfrac{\lambda^2 - a^2 - b^2 }{2ab} \right) - \alpha
$$

Lorsque $\theta=0$, on a $\lambda_0 = \sqrt{d^2 + (c+b)^2}$.


Notons $\gamma$ la position angulaire du moteur et $p$ le pas de la liaison hélicoïdale. On a donc $\lambda = \lambda_0 - p \dfrac{\gamma}{2\pi}=\sqrt{d^2 + (c+b)^2} - p \dfrac{\gamma}{2\pi}$ .

Au final, 
$$
\theta = \arccos \left(\dfrac{\left( \sqrt{d^2 + (c+b)^2} - p \dfrac{\gamma}{2\pi}\right)^2 - a^2 - b^2 }{2ab} \right) - \arctan\left( \dfrac{d}{c}\right)
$$

\subsection{Détermination de la loi en vitesse}
%-u'\rsqcine(1-u²)

On a :
$$
\dot{\theta} = -\dfrac{\dfrac{2\left( \sqrt{d^2 + (c+b)^2} - p \dfrac{\gamma}{2\pi}\right) \left( - p \dfrac{\dot{\gamma}}{2\pi}\right)  }{2ab} }{\sqrt{1-\left(\dfrac{\left( \sqrt{d^2 + (c+b)^2} - p \dfrac{\gamma}{2\pi}\right)^2 - a^2 - b^2 }{2ab} \right)^2}}
$$

\begin{thebibliography}{2}
\bibitem{xx}{xx}
\end{thebibliography}
\end{document}


