\documentclass[10pt]{article}
\input{style/coursHeadings}
\input{style/programHeadings}
\input{style/macros_SII}
\input{style/macros_Titres}
\input{style/macros_Frames}

%Si le boolen xp est vrai : compilation pour xabi
%Sinon compilation Damien
\newboolean{xp}
\setboolean{xp}{true}

\newboolean{prof}
\setboolean{prof}{true}

\usepackage[%
    pdftitle={Étude complète de la capsuleuse},
    pdfauthor={Xavier Pessoles},
    colorlinks=true,
    linkcolor=blue,
    citecolor=magenta]{hyperref}


\def\discipline{Sciences Industrielles de l'Ingénieur}
\def\xxtitre{\ifthenelse{\boolean{xp}}{
Étude des systèmes de laboratoire}{
Chapitre  -- }}

\def\xxsoustitre{\ifthenelse{\boolean{xp}}{
Capsuleuse de bocaux}{
Partie  -- }}

\def\xxauteur{\ifthenelse{\boolean{xp}}{
Xavier \textsc{Pessoles}}{
Damien \textsc{Iceta} \\ Xavier \textsc{Pessoles}}}

\def\xxpied{\ifthenelse{\boolean{xp}}{
Étude des systèmes de laboratoire\\
Capsuleuse de bocaux}{
\xxtitre}}

\def\xxcathegorie{\ifthenelse{\boolean{xp}}{
2013 -- 2014 \\
Xavier \textsc{Pessoles}}{
Informatique - Cours}}





%---------------------------------------------------------------------------


\begin{document}

\ifthenelse{\boolean{xp}}{\input{style/enteteXP}}{\input{style/enteteDI}}

\begin{minipage}[b]{.3\linewidth}
\begin{center}
\includegraphics[width=.95\linewidth]{images/capsuleuse_ph}

\textit{Système pédagogique}
\end{center}
\end{minipage} \hfill
\begin{minipage}[b]{.3\linewidth}
\begin{center}
\includegraphics[width=.95\linewidth]{images/capsuleuse_3d}

\textit{Représentation 3D du système}
\end{center}
\end{minipage} \hfill
\begin{minipage}[b]{.3\linewidth}
\begin{center}
\includegraphics[width=.75\linewidth]{images/CroixMalte_3d}

\textit{Représentation 3D de la Croix de Malte}
\end{center}
\end{minipage}



\setlength{\parskip}{0ex plus 0.2ex minus 0ex}
 \renewcommand{\contentsname}{}
 \renewcommand{\baselinestretch}{1}

\tableofcontents

 \renewcommand{\baselinestretch}{1.2}
\setlength{\parskip}{2ex plus 0.5ex minus 0.2ex}



\section{Modélisation cinématique de la Croix de Malte}
\subsection{Modélisation de la transmission}
Afin de modéliser le système à croix de malte, on propose le schéma cinématique ci-dessous. 

On note :
\begin{itemize}
\item $\mathcal{R}=\left( O,\vect{x_0},\vect{y_0},\vect{z_0}\right)$ le repère lié au bâti $S_0$. On note $\vect{OB}=-L\vect{x_0}$ avec $L = 145\; mm$\footnote{Vérifier les valeurs numériques.};
\item $\mathcal{R}_1=\left( O,\vect{x_1},\vect{y_1},\vect{z_1}\right)$ le repère lié à l'arbre $S_1$. On pose $\vect{OA}=R\vect{y_1}$  avec $R =141\;mm$ et $\alpha = \left( \vect{x_0}, \vect{x_1}\right)$. L'arbre $S_1$ est lié au motoréducteur de la capsuleuse. On a : $\dot{\alpha} = 10\;tr/min$;
\item  $\mathcal{R}_2=\left( B,\vect{x_2},\vect{y_2},\vect{z_2}\right)$ le repère lié à l'arbre $S_2$. On pose $\vect{BA}=\lambda(t)\vect{x_2}$,  $\vect{AI}=r\vect{y_2}$ et $\beta = \left( \vect{x_0}, \vect{x_2}\right)$;
\end{itemize}

\begin{center}
 \includegraphics[width=.7\textwidth]{images/schema1}
\end{center}


\begin{center}
 \includegraphics[width=.4\textwidth]{images/param1}
\end{center}
\subsection{Détermination de la loi Entrée / Sortie}

\subsubsection{Loi Entrée / Sortie -- Position angulaire}
On a :


$$ \vect{OA} + \vect{AB} + \vect{BO} = \vect{0} \Longleftrightarrow 
R\vect{y_1} - \lambda(t)\vect{x_2} + L \vect{x_0} = \vect{0}
$$
En projetant sur $\vect{x_0}$ et $\vect{y_0}$ on a :
$$
\left\{
\begin{array}{l}
- R \sin \alpha(t) - \lambda(t) \cos\beta(t) + L = 0\\
R\cos\alpha(t) - \lambda(t)\sin \beta(t)=0
\end{array}
\right.
$$
Suivant le cas, on peut donc avoir $\alpha$ en fonction de $\beta$ ou $\lambda$ en fonction de $\alpha$ ou $\beta$ :
$$
\tan \beta = \dfrac{R\cos\alpha}{L-R\sin\alpha}
$$

$$
\lambda(t)^2 = R^2 + L^2 - 2RL \sin\alpha
$$

\subsubsection{Loi Entrée / Sortie -- Vitesse angulaire}

On peut calculer : 
$$
\dot{\beta} = \dfrac{R^2\dot{\alpha} - LR\dot{\alpha}\sin\alpha }{L^2-2RL\sin\alpha + R^2}
$$


\subsection{Étude cinématique de la Croix de Malte}
\subsubsection{Détermination de $\{\mathcal{V}(S_1/S_0)\}$}
$$\{\mathcal{V}(S_1/S_0)\} = 
\left\{
\begin{array}{l}
\vect{\Omega(S_1/S_0)} = \dot{\alpha}\vect{z_0} \\
\vect{V(O,S_1/S_0)} = \vect{0}
\end{array}
\right\}_O =
\left\{
\begin{array}{l}
\vect{\Omega(S_1/S_0)} = \dot{\alpha}\vect{z_0} \\
\vect{V(I,S_1/S_0)} = \vect{IO} \wedge \dot{\alpha}\vect{z_0}
\end{array}
\right\}_I
$$

$$
\vect{V(I,S_1/S_0)} =\left( -R\vect{y_1} - r \vect{y_2} \right) \wedge \dot{\alpha}\vect{z_0} =- R\dot{\alpha}\vect{x_1} - r\dot{\alpha}\vect{x_2}
$$
$$\{\mathcal{V}(S_1/S_0)\} = 
\left\{
\begin{array}{l}
\vect{\Omega(S_1/S_0)} = \dot{\alpha}\vect{z_0} \\
\vect{V(I,S_1/S_0)} = -R\dot{\alpha}\vect{x_1} - r\dot{\alpha}\vect{x_2}
\end{array}
\right\}_I
$$

\subsubsection{Détermination de $\{\mathcal{V}(S_2/S_0)\}$}
$$\{\mathcal{V}(S_2/S_0)\} = 
\left\{
\begin{array}{l}
\vect{\Omega(S_2/S_0)} = \dot{\beta}\vect{z_0} \\
\vect{V(B,S_2/S_0)} = \vect{0}
\end{array}
\right\}_B =
\left\{
\begin{array}{l}
\vect{\Omega(S_2/S_0)} = \dot{\beta}\vect{z_0} \\
\vect{V(I,S_2/S_0)} = \vect{IB} \wedge \dot{\beta}\vect{z_0}
\end{array}
\right\}_I
$$
$$
\vect{V(I,S_2/S_0)} =\left( -\lambda(t)\vect{x_2} - r \vect{y_2} \right) \wedge \dot{\beta}\vect{z_0} = \lambda(t) \dot{\beta}\vect{y_2} - r\dot{\beta}\vect{x_2}
$$

$$\{\mathcal{V}(S_2/S_0)\} = 
\left\{
\begin{array}{l}
\vect{\Omega(S_2/S_0)} = \dot{\beta}\vect{z_0} \\
\vect{V(I,S_2/S_0)} = \lambda(t) \dot{\beta}\vect{y_2} - r\dot{\beta}\vect{x_2}
\end{array}
\right\}_I
$$

\subsubsection{Détermination de $\{\mathcal{V}(S_2/S_1)\}$}

D'après la composition du torseur cinématique on a :
$$
\{ \mathcal{V}(S_2/S_1)\} = \{ \mathcal{V}(S_2/S_0)\} + \{ \mathcal{V}(S_0/S_1)\} 
\Longleftrightarrow \{ \mathcal{V}(S_2/S_1)\} = \{ \mathcal{V}(S_2/S_0)\} - \{ \mathcal{V}(S_1/S_0)\} 
$$
On a donc : 
$$
\{ \mathcal{V}(S_2/S_1)\} =
\left\{
\begin{array}{l}
\vect{\Omega(S_2/S_1)} = \vect{\Omega(S_2/S_0)} - \vect{\Omega(S_1/S_0)}  = \left(\dot{\beta}-\dot{\alpha} \right)\vect{z_0} \\
\vect{V(I,S_2/S_1)} = \vect{V(I,S_2/S_0)} - \vect{V(I,S_1/S_0)} 
=
 \lambda(t) \dot{\beta}\vect{y_2} - r\dot{\beta}\vect{x_2} +
 R\dot{\alpha}\vect{x_1} + r\dot{\alpha}\vect{x_2}
\end{array}
\right\}_I
$$

%$$
%\{ \mathcal{V}(S_2/S_1)\} =
%\left\{
%\begin{array}{l}
%\vect{\Omega(S_2/S_1)} = \left(\dot{\beta}-\dot{\alpha} \right)\vect{z_0} \\
%\vect{V(I,S_2/S_1)} =
 %-\lambda(t) \dot{\beta}\vect{x_2}  -  R\dot{\alpha}\vect{y_1} 
%\end{array}
%\right\}_I
%$$

$$ \vect{x_1} 
=\cos(\alpha-\beta)\vect{x_2} + \sin(\alpha-\beta)\vect{y_2}
$$
D'où :
$$
\vect{V(I,S_2/S_1)} =
 \lambda(t) \dot{\beta}\vect{y_2} - r\dot{\beta}\vect{x_2} +
 R\dot{\alpha}\cos(\alpha-\beta)\vect{x_2} +  R\dot{\alpha}\sin(\alpha-\beta)\vect{y_2} + r\dot{\alpha}\vect{x_2}
=\left[
\begin{array}{l}
- r\dot{\beta}+ R\dot{\alpha}\cos(\alpha-\beta) + r\dot{\alpha}\\
 \lambda(t) \dot{\beta}+  R\dot{\alpha}\sin(\alpha-\beta)\\
0\\
\end{array}
\right]_{\mathcal{R}_2}
$$


Nécessairement, la vitesse de glissement appartient au plan tangent au contact. On a donc :
$$
\left\{
\begin{array}{l}
- r\dot{\beta}+ R\dot{\alpha}\cos(\alpha-\beta)  +r\dot{\alpha} = \dot{\lambda}\\
 \lambda(t) \dot{\beta}+  R\dot{\alpha}\sin(\alpha-\beta)=0\\
\end{array}
\right.
$$



\subsection{Détermination de la vitesse du galet}

\begin{minipage}[c]{.4\linewidth}
On considère maintenant l'existence d'un galet $S_3$ en bout de de l'arbre $S_1$. On fait l'hypothèse que le galet roule sans glisser dans le $S_2$. $S_3$ et $S_1$ sont en liaison pivot d'axe $\vect{z_0}$ et de centre $A$.

Le galet a un diamètre extérieur de $16\;mm$. D'après la documentation constructeur, la vitesse de rotation du galet ne doit pas dépasser les $5000\; tr/min$.
\end{minipage} \hfill
\begin{minipage}[c]{.55\linewidth}
\begin{center}
 \includegraphics[width=\textwidth]{images/galet}
\end{center}
\end{minipage}

\begin{center}
 \includegraphics[width=.9\textwidth]{images/schema2}
\end{center}


On fait l'hypothèse de roulement sans glissement selon laquelle la vitesse est nulle entre le galet et la croix de Malte est nulle au point $I$ :
$$ 
\vect{V(I,S_3/S_2)} = \vect{0}
$$


Malgré l'introduction du galet, la position du point $I$ reste inchangée.

Il faut identifier le torseur $\{\mathcal{V}(S_3/S_2)\}$. 
Pour cela, la composition des vitesses donne :
$$
\{\mathcal{V}(S_3/S_2)\} = \{\mathcal{V}(S_3/S_1)\} + \{\mathcal{V}(S_1/S_2)\}  
$$


Au point $I$ on connaît déjà $\{\mathcal{V}(S_1/S_2)\}$.

Calculons $\{\mathcal{V}(S_3/S_1)\}$:

$$
\{\mathcal{V}(S_3/S_1)\}
= 
\left\{
\begin{array}{l}
\vect{\Omega(S_3/S_1)} = \dot{\gamma}\vect{z_0} \\
\vect{V(A,S_3/S_1)} = \vect{0}
\end{array}
\right\}_A =
\left\{
\begin{array}{l}
\vect{\Omega(S_3/S_1)} = \dot{\gamma}\vect{z_0} \\
\vect{V(I,S_3/S_1)} = \vect{IA} \wedge \dot{\gamma}\vect{z_0}
= -r\vect{y_2} \wedge \dot{\gamma}\vect{z_0} = -r \dot{\gamma}\vect{x_2}
\end{array}
\right\}_I
$$

On a donc :
$$
\vect{V(I,S_3/S_2)} = \vect{V(I,S_3/S_1)} + \vect{V(I,S_1/S_2)} 
$$
$$
\vect{V(I,S_3/S_2)} = -r \dot{\gamma}\vect{x_2} 
+\left(-r\dot{\beta}+ R\dot{\alpha}\cos(\alpha-\beta) + r\dot{\alpha}\right)\vect{x_2}
- \left(\lambda(t) \dot{\beta}+  R\dot{\alpha}\sin(\alpha-\beta)\right)\vect{y_2}
$$


$$
\vect{V(I,S_3/S_2)} =\left[ 
\begin{array}{l}
 -r \dot{\gamma}+\left(-r\dot{\beta}+ R\dot{\alpha}\cos(\alpha-\beta) + r\dot{\alpha}\right)\\
- \left(\lambda(t) \dot{\beta}+  R\dot{\alpha}\sin(\alpha-\beta)\right)\\
0\\
\end{array}
\right]_{\mathcal{R}_2}  
$$


%De même que précédemment, les solides étant indéformable, la projection de la vitesse sur $\vect{y_2}$ est nulle. 

D'après l'hypothèse de roulement sans glissement, on a :
$$ 
\vect{V(I,S_3/S_2)} = \vect{0} \Longrightarrow  \dot{\gamma}=-\dfrac{-r\dot{\beta}+ R\dot{\alpha}\cos(\alpha-\beta) + r\dot{\alpha}}{r}
$$



\begin{thebibliography}{2}
\bibitem{xx}{xx}
\end{thebibliography}
\end{document}


